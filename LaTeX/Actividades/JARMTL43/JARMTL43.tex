%Plantilla para taller de LaTeX

\documentclass[12pt]{article}
\usepackage[spanish]{babel}
\usepackage[utf8]{inputenc}
\usepackage{amssymb}
\usepackage{graphicx}

\title{Ejemplo para insertar gráficos y figuras en \LaTeXe}
\author{Jesus Adrian Rosas Martinez}
\date{\small{\today}}

\begin{document}

\maketitle %Encabezado

Este es un pequeño ejemplo que muestra los efectos producidos por diversas opciones del entorno figure. Comencemos con el comando $\backslash$\texttt{listoffigures} que produce el siguiente índice:

\listoffigures

Este es el gráfico en su tamaño original:
\begin{figure}
\centering
\includegraphics[keepaspectratio]{LaTeXlion2.png}
\end{figure}

\begin{figure}
\centering
\includegraphics[width=\textwidth]{LaTeXlion2.png}
\caption{Gráfico con textwidth}
\label{fig:1}
\end{figure}

\begin{figure}
\centering
\includegraphics[width=1.0cm]{LaTeXlion2.png}
\caption{Gráfico con width=1.00 cm}
\label{fig:2}
\end{figure}

\begin{figure}
\centering
\includegraphics[width=1.0cm, height=3.0cm]{LaTeXlion2.png}
\caption{Gráfico con width=1.00 cm y height=3.00 cm}
\label{fig:3}
\end{figure}

\begin{figure}
\centering
\includegraphics[width=1.0cm, height=3.0cm, keepaspectratio]{LaTeXlion2.png}
\caption{Gráfico con width=1.00 cm, height=3.00 cm y keepaspectratio}
\label{fig:4}
\end{figure}

\begin{figure}
\centering
\includegraphics[width=1.0cm, height=1.0cm, draft]{LaTeXlion2.png}
\caption{Gráfico con width=1.00 cm, height=1.00 cm y draft}
\label{fig:5}
\end{figure}

\begin{figure}
\centering
\includegraphics[scale=0.75]{LaTeXlion2.png}
\caption{Gráfico con scale=0.75}
\label{fig:6}
\end{figure}

\begin{figure}
\centering
\includegraphics[width=3.0cm, height=3.0cm, angle=45]{LaTeXlion2.png}
\caption{Gráfico con width=3.00 cm, height=3.00 cm y angle=45}
\label{fig:7}
\end{figure}

\begin{center}
\large{\textsc{EXPLICACIÓN}}\\
\end{center}
\begin{itemize}
\item \textbf{width}: La opción width, por si sola, modifica el tamaño de la imagen (ancho y alto), dependiendo la medida que se desea.
\item \textbf{height}: La opción height modifica el tamaño de la imagen a lo largo.
\item \textbf{$\backslash$textwidth}: El parámetro $\backslash$textwidth modifica el tamaño de la imagen respetando el margen del documento.
\item \textbf{keepaspectratio}: Modifica el tamaño de la imagen, sin deformarla, por si solo coloca la imagen a su tamaño original ya que es de tipo lógico (no importa si se escribe o se omite).
\item \textbf{scale}: Escala una imagen, dependiendo el tamaño que se desea.
\item \textbf{draft}: Coloca un margen alrededor de la imagen sin mostrarla y solo coloca el nombre que tiene.
\item \textbf{angle}: Rota la imagen dependiendo el numero de grados que se desea.
\end{itemize}


\end{document}