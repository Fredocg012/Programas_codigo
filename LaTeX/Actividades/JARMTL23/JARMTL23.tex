\documentclass[12pt]{article}
\usepackage[spanish]{babel}
\usepackage[utf8]{inputenc}

\title{TL23: La estructura de un documento en LaTex}
\author{Jes\'us Adri\'an Rosas Mart\'inez}
\date{\small{\today}}

\begin{document}

\maketitle %Encabezado

\abstract{LaTex es muy utilizado para la composición de artículos académicos, tesis y libros técnicos, dado a que la calidad tipográfica de los documentos realizados es adecuada a las necesidades de una editorial científica de primera línea.}

\tableofcontents %Indice

\section{LaTeX} %sección
\label{sec:latex} %referencia cruzada
\subsection{¿Qué es LaTeX?} %subsección
LaTex es un sistema de composición de textos, orientado a la creación de documentos escritos que presenten una alta calidad tipográfica.

Por sus características y posibilidades, es usado de forma especialmente intensa en la generación de artículos y libros científicos que incluyen, entre otros elementos, expresiones matemáticas.

\subsection{¿Por qué usar LaTeX?} %subsección
Una de las ventajas de LaTeX es la calidad profesional de los documentos que puedes generar. \textsl{\footnotesize{La sección \ref{sec:latex} fundamenta esta ventaja}}.

LaTeX, además, realiza de manera automática muchas tareas que de otro modo podrían resultar tediosas como: numerar capítulos y figuras, incluir y organizar la bibliografía adecuada, mantener índices y referencias cruzadas, etcétera.

\section{Generar documentos PDF con LaTeX} %sección
Para generar un documento pdf con LaTeX se requiere partir de un archivo de extensión .tex. Este se produce desde un editor y consta de las siguientes partes:

\begin{enumerate} %lista enumerada
\item \textbf{\Large{Preámbulo:}} En el se indica la clase de documento que se producirá, los paquetes que se utilizarán, comandos para la configuración de la apariencia del documento, etcétera.
\item \textbf{\Large{Cuerpo del documento:}} En el se vierten los contenidos del texto que se desea producir.
\end{enumerate}

Una vez constituido el archivo .tex por el \textit{\small{preámbulo}} y el \textit{\small{cuerpo del documento}}, se debe compilar \emph{(procesar)} para generar el archivo PDF.


\section{Composición de documentos .tex}
\subsection{Clases de documentos en LaTeX}
En LaTeX existen 5 tipos de clase para crear documentos estándar y son las siguientes: 

\begin{center}
\large{\textsc{Tipos de clases}}
\end{center}

\begin{itemize}
\item \textbf{article:} Artículos para revistas científicas, tutoriales, etcétera.
\item \textbf{report:} Textos largos como tesis, reportes, etcétera.
\item \textbf{book:} Libros o documentos con una estructura similar.
\item \textbf{letter:} Cartas.
\item \textbf{slides:} Transparencias.
\end{itemize}

\section{Fuentes de consulta}

\begin{flushright}
Taller de LaTeX

Página web: nokyotsu.com/latex/guia.html

Página web: desarrolloweb.dlsi.ua.es/cursos/2015/herramientas-investigacion/que-es-latex
\end{flushright}

\begin{flushleft}
\scriptsize{\texttt{Nota: Este documento\\aborda temas\\vistos anteriormente\\en el taller.}}
\end{flushleft}

\footnote{UNAM}

\end{document}