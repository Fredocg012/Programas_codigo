%Plantilla para taller de LaTeX

\documentclass[12pt]{article}
\usepackage[spanish]{babel}
\usepackage[utf8]{inputenc}

\title{Administrador de referencias \emph{JabRef}}
\author{Jesus Adrian Rosas Martinez}
\date{\small{\today}}

\begin{document}

\maketitle %Encabezado

\LaTeX es un sistema de composición de textos orientado a la creación de documentos escritos que presenten una alta calidad tipográfica. Por sus características y posibilidades, es usado de forma especialmente intensa en la generación de artículos (véase \cite{li1975bases}) y reportes científicos (véase \cite{zuber1959hydrodynamic}) que incluyen, entre otros elementos, expresiones matemáticas.

Es muy utilizado para la composición de trabajos técnicos (véase \cite{gianturco2012migracion}), publicaciones científicas (véase \cite{alarcon2010publicaciones}) y tesis, dado que la calidad tipográfica de los documentos realizados en \LaTeX, se le considera adecuada a las necesidades de una editorial científica de primera linea.

Finalmente, se recomienda consultar la guía rápida para el nuevo usuario de \LaTeX (véase \cite{martel2004guia}).

\bibliographystyle{alpha}
\bibliography{referencias1}

\end{document}