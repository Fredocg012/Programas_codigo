%Plantilla para elaborar presentaciones con LaTeX Beamer

%\documentclass[spanish]{beamer}
\documentclass[xcolor=pdftex,dvinames,table]{beamer}
\usepackage[spanish]{babel}
%\usepackage[latin1]{inputenc}
\usepackage[utf8]{inputenc}
%%Para los acentos
\usetheme{Warsaw}

\title{Elaboración de presentaciones con la clase \LaTeX Beamer}
\author{Jesus Adrian Rosas Martinez}
\date{\today}

\begin{document}
\frame{\titlepage}
\begin{frame}
\frametitle{Índice} 
\tableofcontents %Índice de la presentación
\end{frame}
\section{Listas y viñetas}%Bloque para crear diapositivas
\begin{frame}
\frametitle{Listas y viñetas}
Las listas y viñetas se producen con los mismos comandos que se emplean para este fin en los documentos creados con \LaTeX.\\
\begin{enumerate}
\item Un elemento
\item Otro elemento
\end{enumerate}
\begin{itemize}
	\item [(i)] Un elemento
	\item [(ii)] Otro elemento\\
	\item Otro \ldots
	\item El último
\end{itemize}
\end{frame}%Fin de bloque

\section{Integración de imágenes}%Bloque para crear diapositivas
\begin{frame}
\frametitle{Integración de imágenes}
No se recomienda el uso de figuras en las presentaciones. Deben incluirse sólo imágenes que aporten algo al material o bien sirvan de soporte. He aquí un ejemplo:
\begin{center}
\includegraphics{LaTeXlion2.png}
\end{center}
\end{frame}

\section{Tablas con color}%Bloque para crear diapositivas
\begin{frame}
\frametitle{Tablas con color}
Las tablas que se pueden integrar en una presentación con Beamer, pueden o no tener color. He aquí un ejemplo:\bigskip

\rowcolors{1}{red!20}{red!15}
\begin{tabular}{ll}
\textbf{Término}&\textbf{Descripción}\\
\hline
\LaTeX perto& Usuario avanzado de \LaTeX\\
\LaTeX plorador& Usuario principiante de \LaTeX\\
\LaTeX iliado& Quien conoce \LaTeX pero se niega a usarlo\\
\LaTeX cluido& Quien no conoce \LaTeX\\
\hline
\end{tabular}
\end{frame}

\section{División de diapositivas}%Bloque para crear diapositivas
\begin{frame}
\frametitle{División de diapositivas}
Las diapositivas pueden dividirse en columnas. Aquí se presenta esta diapositiva dividida en dos columnas de 2 pulgadas de ancho: \\
\begin{columns}

\column{2in}
\begin{flushleft}
\LaTeX es un sistema de composición de textos, orientado especialmente a la creación de libros, documentos científicos y técnicos que contengan fórmulas matemáticas.
\end{flushleft}

\column{2in}
\begin{center}
\includegraphics[height=1.5in]{LaTeXlion1.png}
\end{center}
\end{columns}
\end{frame}


\section{Matemáticas ``encapsuladas''}%Bloque para crear diapositivas
\begin{frame}
\frametitle{Matemáticas ``encapsuladas''}
En ocasiones es conveniente resaltar definiciones, axiomas, proposiciones, teoremas, corolarios y demostraciones; como en el siguiente ejemplo: \\
\begin{block}{Teorema de Pitágoras}
Sean $a$ y $b$ los catetos de un triángulo rectángulo; y $c$, la hipotenusa. Entonces, se satisface que\\
\begin{center}
$c^{2}=a^{2}+b^{2}$.
\end{center}
\end{block}
\end{frame}


\section{Referencias bibliográficas}%Bloque para crear diapositivas
\begin{frame}
\frametitle{Referencias bibliográficas}
Las referencias pueden insertarse como en cualquier documento de \LaTeX.

\begin{thebibliography}{1}
\bibitem{Tantau}
Till Tantau. \emph{Users Guide to the Beamer Class,
v 3.01}. En \url{http://latex-beamer.sourceforge.net}.
Ultima consulta: 16 de junio de 2009.
\end{thebibliography}

\end{frame}

\begin{frame}

\begin{thebibliography}{1}
\bibitem{zuber1959hydrodynamic}
Novak Zuber.
\newblock Hydrodynamic aspects of boiling heat transfer (thesis).
\newblock Technical report, Ramo-Wooldridge Corp., Los Angeles, CA (United
  States); Univ. of California, Los Angeles, CA (United States), 1959.
\end{thebibliography}

\begin{thebibliography}{1}
\bibitem{martel2004guia}
Eugenio~Fedriani Martel.
\newblock {\em Gu{\'\i}a r{\'a}pida para el nuevo usuario de LATEX}.
\newblock Juan Carlos Mart{\'\i}nez Coll, 2004.
\end{thebibliography}

\end{frame}


\end{document}