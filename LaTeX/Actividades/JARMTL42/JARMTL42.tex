%Plantilla para taller de LaTeX

\documentclass[12pt]{article}
\usepackage[spanish]{babel}
\usepackage[utf8]{inputenc}
\usepackage{longtable}
\usepackage{float}
\usepackage{lscape}

\addto\captionsspanish{
\def\listtablename{\'Indice de tablas}%
\def\tablename{Tabla}}

\title{Tablas con \LaTeX}
\author{Jesus Adrian Rosas Martinez}
\date{\small{\today}}

\begin{document}

\maketitle %Encabezado

A continuación, se presentan algunos ejemplos de tablas creadas con \LaTeX. Los entornos \{longtable\} y \{landscape\} se distribuyen con MikTeX, por lo que no es necesario descargarlos de ningún repositorio.

\listoftables

\begin{table} %Tabla 1
\centering
\caption{Tabla \emph{con} especificación de ancho de columna}
\begin{tabular}{p{2.5cm}p{8cm}}
\\
\hline
\hline
\textbf{Término} & \textbf{Descripción}\\
\hline
\LaTeX perto & Usuario avanzado de \LaTeXe\ capacitado para la obtención y empleo de paquetes, herramientas y/o aplicaciones desarrolladas para lograr sus propósitos al realizar documentos con \LaTeX.\\
\hline
\LaTeX plorador & Usuario principiante de \LaTeXe\ cuya ocupación principal consiste en la recopilación de fuentes de información sobre el uso de \LaTeX.\\
\hline
\LaTeX iliado & Usuario de procesadores de texto convencionales que insiste en que no necesita \LaTeX\ para nada, o bien, en que éste no le es de utilidad.\\
\hline
\LaTeX cluido & Usuario de procesadores de texto convencionales que desconoce la existencia de \LaTeX.\\
\hline
\hline
\end{tabular}
\label{tab:1}
\end{table}

\begin{table}[H]%Tabla 2
\centering
\caption{Tabla \emph{con} encabezados centrados}
\begin{tabular}{p{2.5cm}p{8cm}}
\\
\hline
\hline
\multicolumn{1}{c}{\textbf{Término}} & 
\multicolumn{1}{c}{\textbf{Descripción}}\\
\hline
\LaTeX perto & Usuario avanzado de \LaTeXe\ capacitado para la obtención y empleo de paquetes, herramientas y/o aplicaciones desarrolladas para lograr sus propósitos al realizar documentos con \LaTeX.\\
\hline
\LaTeX plorador & Usuario principiante de \LaTeXe\ cuya ocupación principal consiste en la recopilación de fuentes de información sobre el uso de \LaTeX.\\
\hline
\LaTeX iliado & Usuario de procesadores de texto convencionales que insiste en que no necesita \LaTeX\ para nada, o bien, en que éste no le es de utilidad.\\
\hline
\LaTeX cluido & Usuario de procesadores de texto convencionales que desconoce la existencia de \LaTeX.\\
\hline
\hline
\end{tabular}
\label{tab:2}
\end{table}

\begin{table} %Tabla 3
\centering
\footnotesize
\caption{Tabla \emph{con} tamaño de fuente igual al de los pies de página}
\begin{tabular}{p{2.5cm}p{8cm}}
\\
\hline
\hline
\textbf{Término} & \textbf{Descripción}\\
\hline
\LaTeX perto & Usuario avanzado de \LaTeXe\ capacitado para la obtención y empleo de paquetes, herramientas y/o aplicaciones desarrolladas para lograr sus propósitos al realizar documentos con \LaTeX.\\
\hline
\LaTeX plorador & Usuario principiante de \LaTeXe\ cuya ocupación principal consiste en la recopilación de fuentes de información sobre el uso de \LaTeX.\\
\hline
\LaTeX iliado & Usuario de procesadores de texto convencionales que insiste en que no necesita \LaTeX\ para nada, o bien, en que éste no le es de utilidad.\\
\hline
\LaTeX cluido & Usuario de procesadores de texto convencionales que desconoce la existencia de \LaTeX.\\
\hline
\hline
\end{tabular}
\label{tab:3}
\end{table}

\centering %Tabla 4
\normalsize
\begin{longtable}{p{2.5cm}p{8cm}}
\\
\caption{Tabla \emph{que} ocupa más de una página}
\\
\hline
\hline
\multicolumn{1}{l}{\textbf{Término}} &
\multicolumn{1}{l}{\textbf{Descripción}} \\[0.5ex] \hline
\\[-1.8ex]
\endfirsthead
\multicolumn{2}{c}{{\tablename} \thetable{} -- Continuación}
\\[0.5ex]
\hline \hline \\[-2ex]
\multicolumn{1}{l}{\textbf{Término}} &
\multicolumn{1}{l}{\textbf{Descripción}} \\[0.5ex]
\\[-1.8ex]
\endhead
\multicolumn{2}{r}{{Continúa en la siguiente página\ldots}}
\\
\endfoot
\\[-1.8ex]
\endlastfoot
\LaTeX perto & Usuario avanzado de \LaTeXe\ capacitado para la obtención y empleo de paquetes, herramientas y/o aplicaciones desarrolladas para lograr sus propósitos al realizar documentos con \LaTeX.\\
\hline
\LaTeX perto & Usuario avanzado de \LaTeXe\ capacitado para la obtención y empleo de paquetes, herramientas y/o aplicaciones desarrolladas para lograr sus propósitos al realizar documentos con \LaTeX.\\
\hline
\LaTeX perto & Usuario avanzado de \LaTeXe\ capacitado para la obtención y empleo de paquetes, herramientas y/o aplicaciones desarrolladas para lograr sus propósitos al realizar documentos con \LaTeX.\\
\hline
\LaTeX plorador & Usuario principiante de \LaTeXe\ cuya ocupación principal consiste en la recopilación de fuentes de información sobre el uso de \LaTeX.\\
\hline
\LaTeX plorador & Usuario principiante de \LaTeXe\ cuya ocupación principal consiste en la recopilación de fuentes de información sobre el uso de \LaTeX.\\
\hline
\LaTeX plorador & Usuario principiante de \LaTeXe\ cuya ocupación principal consiste en la recopilación de fuentes de información sobre el uso de \LaTeX.\\
\hline
\LaTeX iliado & Usuario de procesadores de texto convencionales que insiste en que no necesita \LaTeX\ para nada, o bien, en que éste no le es de utilidad.\\
\hline
\LaTeX iliado & Usuario de procesadores de texto convencionales que insiste en que no necesita \LaTeX\ para nada, o bien, en que éste no le es de utilidad.\\
\hline
\LaTeX iliado & Usuario de procesadores de texto convencionales que insiste en que no necesita \LaTeX\ para nada, o bien, en que éste no le es de utilidad.\\
\hline
\LaTeX cluido & Usuario de procesadores de texto convencionales que desconoce la existencia de \LaTeX.\\
\hline
\LaTeX cluido & Usuario de procesadores de texto convencionales que desconoce la existencia de \LaTeX.\\
\hline
\LaTeX cluido & Usuario de procesadores de texto convencionales que desconoce la existencia de \LaTeX.\\
\hline
\hline
\label{tab:4}
\end{longtable}

\begin{landscape} %Tabla 5
\centering
\normalsize
\begin{longtable}{p{2.5cm}p{8cm}}
\\
\caption[Tabla \emph{``acostada''}]{Tabla \emph{``acostada''}}
\\
\hline
\hline
\multicolumn{1}{l}{\textbf{Término}} &
\multicolumn{1}{l}{\textbf{Descripción}} \\[0.5ex] \hline
\\[-1.8ex]
\endfirsthead
\multicolumn{2}{c}{{\tablename} \thetable{} -- Continuación}
\\[0.5ex]
\hline \hline \\[-2ex]
\multicolumn{1}{l}{\textbf{Término}} &
\multicolumn{1}{l}{\textbf{Descripción}} \\[0.5ex]
\\[-1.8ex]
\endhead
\multicolumn{2}{r}{{Continúa en la siguiente página\ldots}}
\\
\endfoot
\\[-1.8ex]
\endlastfoot
\LaTeX perto & Usuario avanzado de \LaTeXe\ capacitado para la obtención y empleo de paquetes, herramientas y/o aplicaciones desarrolladas para lograr sus propósitos al realizar documentos con \LaTeX.\\
\hline
\LaTeX perto & Usuario avanzado de \LaTeXe\ capacitado para la obtención y empleo de paquetes, herramientas y/o aplicaciones desarrolladas para lograr sus propósitos al realizar documentos con \LaTeX.\\
\hline
\LaTeX perto & Usuario avanzado de \LaTeXe\ capacitado para la obtención y empleo de paquetes, herramientas y/o aplicaciones desarrolladas para lograr sus propósitos al realizar documentos con \LaTeX.\\
\hline
\LaTeX plorador & Usuario principiante de \LaTeXe\ cuya ocupación principal consiste en la recopilación de fuentes de información sobre el uso de \LaTeX.\\
\hline
\LaTeX plorador & Usuario principiante de \LaTeXe\ cuya ocupación principal consiste en la recopilación de fuentes de información sobre el uso de \LaTeX.\\
\hline
\LaTeX plorador & Usuario principiante de \LaTeXe\ cuya ocupación principal consiste en la recopilación de fuentes de información sobre el uso de \LaTeX.\\
\hline
\LaTeX iliado & Usuario de procesadores de texto convencionales que insiste en que no necesita \LaTeX\ para nada, o bien, en que éste no le es de utilidad.\\
\hline
\LaTeX iliado & Usuario de procesadores de texto convencionales que insiste en que no necesita \LaTeX\ para nada, o bien, en que éste no le es de utilidad.\\
\hline
\LaTeX iliado & Usuario de procesadores de texto convencionales que insiste en que no necesita \LaTeX\ para nada, o bien, en que éste no le es de utilidad.\\
\hline
\LaTeX cluido & Usuario de procesadores de texto convencionales que desconoce la existencia de \LaTeX.\\
\hline
\LaTeX cluido & Usuario de procesadores de texto convencionales que desconoce la existencia de \LaTeX.\\
\hline
\LaTeX cluido & Usuario de procesadores de texto convencionales que desconoce la existencia de \LaTeX.\\
\hline
\hline
\label{tab:5}
\end{longtable}
\end{landscape}

\end{document}