%Plantilla para taller de LaTeX

\documentclass[12pt]{article}
\usepackage[spanish]{babel}
\usepackage[utf8]{inputenc}
\usepackage{amssymb}
\usepackage{graphicx}

\title{E. C. Escher. \emph{Manos dibujando.}}
\author{\small{Jes\'us Adri\'an Rosas Mart\'inez}}
\date{\small{\today}}

\begin{document}

\maketitle %Encabezado

La pintura que eh elegido es conocida con el nombre ``Manos dibujando'', creada por E. C. Escher. Esta obra llamó mi atención y me dejó asombrado, antes de analizarla es importante conocer un poco sobre el artista.\\

Maurits Cornelis Escher fué un artista neerlandés conocido por sus grabados xilográficos, sus grabados al mezzotinto y sus dibujos que consisten en figuras imposibles, teselados y mundos imaginarios.  Su obra experimenta con diversos métodos de representar espacios paradójicos que desafían a los modos habituales de representación. \\

Tomando en cuenta algunos rasgos característicos de la obra de arte se afirma que la obra ``Manos dibujando'' cuenta con todas las características (o la mayoría); la obra ya antes citada es una creación artística única e imaginaria, ya que en ella se muestran dos manos dibujando una a la otra, algo verdaderamente confuso pero impresionante. Escher sin duda plasma en la obra su creatividad y talento ya que ambas manos parecieran ser tridimensionales, pero en realidad están hechas en un plano.\\

Escher a través de sus obras nos transmite sin duda sentimientos y diversas sensaciones. Al ver la obra por unos minutos eh sentido como si mis manos fueran las de la obra, me confundo si trato de ver como lo ha hecho o empezado. Es un verdadero reto hacer una copia a mano; aquí se afirma que cualquier obra de arte es irreproducible \\

{En conclusión, cualquier obra de arte nos transmite sentimientos y sensaciones, eso es lo que las hace importantes, valiosas y únicas. Cada artista plasma en sus obras su propia personalidad, inteligencia, talento y creatividad. 

Para determinar si ``algo'' es una obra de arte se deben tomar en cuenta varias características fundamentales, ya que puede que para nosotros sea arte y para los demás no.} \\



\end{document}